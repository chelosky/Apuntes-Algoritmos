%% \union - Example: \union{j \in J}{A_j}
\newcommand{\union}[2]{\underset{#1}\bigcup #2}

%% \inter - like \union, but with \bigcap
\newcommand{\inter}[2]{\underset{#1}\bigcap #2}

%% \nuevoTitulo - Example \nuevoTitulo{Guia 1} -> Guia 1
\newcommand{\nuevoTitulo}[1]{\begin{center}\section*{#1}\end{center}}

%%\renewcommand{\qedsymbol}{$\blacksquare$}

%% Redefine the solution title x3
\renewcommand{\solutiontitle}{\noindent\textbf{Solución:}\par\noindent}

%% \sumatoria - Example \sumatoria{n=1}{\infty} -> sumatoria de n desde 1 a infinito
\renewcommand{\sumatoria}[2]{\sum\limits_{#1}^{#2}}

%% PARA MOSTRARA UNA GENERATRIZ
\renewcommand{\funcGeneratriz}[1]{\mathfrak{S}\left[ #1 \right]}

%% PARA MOSTRAR UN GENERATRIZ INVERSA
\renewcommand{\funcInvGeneratriz}[1]{\mathfrak{S}^{-1}\left[ #1 \right]}
