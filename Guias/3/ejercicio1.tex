\question{
Resolver la ecuación con un cambio de variable:

\begin{align*}
 T(n) &= 2T\left(\frac{n}{3}\right) + \frac{n}{3}\quad, n > 1\\
 T(1) &= 0
\end{align*}
}
\begin{solution}
  En primer lugar debemos determinar que cambio de variable aplicaremos, esto se hace determinando los primeros terminos de la sucesión:
  \begin{center}
    \begin{tabular}{@{}llllll@{}}
        \toprule
        n & 1 & 3 & 9 & 27 \\ \midrule
        T(n) & 0 & 1 & 5 & 19 \\ 
        \bottomrule
    \end{tabular}\\
  \end{center}
  
  Podemos observar que n es múltiplo de 3 ($3^0$ = 1, $3^1$ = 3, $3^2$ = 9,...), por lo tanto el cambio de variable que debemos hacer es:
  \begin{center}
      n = $3^k$, con $k \ge 0$
  \end{center}
  
  Aplicamos el cambio de variable a la ecuación de recurrencia inicial ($T(n) = 2T\left(\frac{n}{3}\right) + \frac{n}{3}$):
  \begin{align*}
      T(3^k) &= 2T\left(\frac{3^k}{3}\right) + \frac{3^k}{3}\\
      T(3^k) &= 2T(3^{k-1}) + 3^{k-1}
  \end{align*}

  Hacemos una nueva sustitución para poder resolver la ecuación y determinamos su condición inicial:
  \begin{align*}
      X_k = T(3^k) \implies X_{k-1} = T(3^{k-1})
  \end{align*}
  \begin{center}
      La condición inicial la determinamos asignando $k = 0$:
      
      Si $k = 0$ $\implies$ $n = 1$ $\implies$ $T(1) = 0$
      
      Por lo tanto la condición inicial queda:
      
      $$X_0 = 0$$
  \end{center}
  
\end{solution}
