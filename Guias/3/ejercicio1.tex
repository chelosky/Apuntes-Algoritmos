\question{
Resolver la ecuación con un cambio de variable:

$T(n) = 2T(n/3) + n/3, n > 1$

$T(1) = 0$
}
\begin{solution}
  En primer lugar debemos determinar que cambio de variable aplicaremos, esto se hace determinando los primeros terminos de la sucesión:
  \begin{center}
    \begin{tabular}{@{}llllll@{}}
        \toprule
        n & 1 & 3 & 9 & 27 \\ \midrule
        T(n) & 0 & 1 & 5 & 19 \\ 
        \bottomrule
    \end{tabular}\\
  \end{center}
  
  Podemos observar que n es múltiplo de 3 ($3^0$ = 1, $3^1$ = 3, $3^2$ = 9,...), por lo tanto el cambio de variable que debemos hacer es:
  \begin{center}
      n = $3^k$, con $k \ge 0$
  \end{center}
  
  Aplicamos el cambio de variable a la ecuación de recurrencia inicial ($T(n) = 2T(n/3) + n/3$):
  
  \begin{center}
      T($3^k$) = 2T($3^k$/3) + $3^k$/3
      
      T($3^k$) = 2T($3^{k-1}$) + $3^{k-1}$
  \end{center}
  
  Hacemos una nueva sustitución para poder resolver la ecuación y determinamos su condición inicial:
  \begin{center}
      $X_k$ = T($3^k$) \implies $X_{k-1}$ = T($3^{k-1}$)
      
      La condición inicial la determinamos asignando k = 0:
      
      Si k = 0 \implies n = 1 \implies T(1) = 0
      
      Por lo tanto la condición inicial queda:
      
      $X_0$ = 0
      
  \end{center}
  
\end{solution}
