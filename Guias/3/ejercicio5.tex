\question{
Resolver:
\begin{center}
$X_{n+2}$ = $X_n$ + 8n + 8, $n > 1$

$X_0$ = 1

$X_1$ = 3
\end{center}
}
\begin{solution}
  En primer lugar, calcularemos dos condiciones de borde más:
  \begin{center}
      $X_2$ = 9, $X_3$ = 19
  \end{center}
  
  Dejaremos a un lado los $X_n$:
  \begin{center}
      $X_{n+2} - X_n$ = 8n + 8
  \end{center}
  
  Aplicaremos 2 veces el operador $\bigtriangleup$ para reducir el polinomio (parte derecha) para que sea 0. Resolvemos el polinomio característico de la ecuación:
  \begin{center}
    $X_{n+2} - X_n = 0$
  
    Polinomio característico: 
    
    $\lambda^2$ - 1 = 0
    
    ($\lambda$ + 1)($\lambda$ - 1) = 0
    
    $\lambda_1$ = -1, $\lambda_2$ = 1
    
    Como aplicamos dos veces el operador $\bigtriangleup$, consideramos las 2 raíces iguales a 1:
    
    $\lambda_3$ = 1, $\lambda_4$ = 1
  \end{center}
  
  La solución general queda:
  \begin{center}
      $X_n = C_1 + C_2 \cdot n + C_3 \cdot n^2 + C_4 \cdot (-1)^n$
  \end{center}
  
  Debemos calcular la solución particular con las 4 condiciones de borde que tenemos:
  \begin{center}
      $X_0 \implies 1 = C_1 + C_4$
      
      $X_1 \implies 3 = C_1 + C_2 + C_3 - C_4$
      
      $X_2 \implies 9 = C_1 + 2C_2 + 4C_3 + C_4$
      
      $X_3 \implies 19 = C_1 + 3C_2 + 9C_3 - C_4$
  \end{center}
  
  Una vez determinadas las soluciones, se debe resolver el sistema de ecuaciones, lo que nos da:
  \begin{center}
      $C_1 = 1, C_2 = 0, C_3 = 2, C_4 = 0$
  \end{center}
  
  Por último, reemplazamos los valores en la solución general:
  \begin{center}
      $X_n = 1 + 0 \cdot n + 2 \cdot n^2 + 0 \cdot (-1)^n$
      
      $\implies X_n = 1 + 2 \cdot n^2 , n \geq 0$
  \end{center}
  
\end{solution}
