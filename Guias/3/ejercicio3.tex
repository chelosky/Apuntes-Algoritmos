\question{
Para el siguiente algortimo, sabiendo que inicialmente se le llama con $Prueba(A, 1, n)$ para $n > 0$, escriba y resuelva una ecuación de recurrencia para la cantidad de sumas realizadas (instrucción $/***/$):

\begin{lstlisting}
// Ejercicio3_Guia3.java
public int Prueba(int[] A,int i,int j) {
    if(i >= j){
        return(A[j]);
    }else{
        int s=0;
        for(int k=i; k <= j/3;k++){
            s = s + A[k]; /***/
        }
        Prueba(A, j/3 +1,(2*j)/3);
        Prueba(A, (2*j)/3 +1,j);
    }
}    
\end{lstlisting}

}
\begin{solution}
Sea $A(n)$ la cantidad de veces que se ejecuta la instrucción $/***/$. Además, se sabe que el valor inicial es $A(1)=0$ dado que si $n=1$ entonces la instrucción $/***/$ nunca se ejecuta.

En terminos generales, el algoritmo ejecutara un bucle for con la instrucción $/***/$ una cantidad de $\frac{n}{3}$ veces. Luego se llama recursivamente la misma función 2 veces para los terminos $A(\frac{n}{3})$.
Con esto la ecuación de recurrencia resultante es:
\begin{align*}
    A(n) &= 2A\left(\frac{n}{3}\right) + \frac{n}{3}\quad, n>1 \\
    A(1)&=0
\end{align*}
Aplicamos un cambio de variable equivalente a:
\begin{align*}
    n &= 3^k  \xrightarrow{} \log_3(n)=k
\end{align*}
Al aplicar dicho cambio de variable en la ecuación de recurrencia inicial se obtiene:
\begin{align*}
    A(3^k) &= 2A(3^{k-1}) + 3^{k-1} \tag{$\spadesuit$}
\end{align*}
Definamos una nueva sustitución tal que:
\begin{align*}
    A(3^k) &= X_k \xrightarrow{con} A(1)=X_0
\end{align*}
Al aplicar dicha en sustitución en la ecuación ($\spadesuit$), tal que:
\begin{align*}
    X_k &= 2X_{k-1} + 3^{k-1} \tag{$\heartsuit$} \\
    X_0 &=0
\end{align*}
Esta ecuación se resuelve aplicando factor sumante $S_k$. Definido como:
\begin{align*}
    S_k &= \frac{1}{2^k}
\end{align*}
Al multiplicar el factor sumantes $S_k$ en ambos lados de la ecuación ($\heartsuit$), se obtiene:
\begin{align*}
    S_k X_k &= 2 S_k X_{k-1} + 3^{k-1} S_k\\
    \frac{X_k}{2^k} &= \frac{X_{k-1}}{2^{k-1}} + \frac{1}{3} \left(\frac{3}{2}\right)^k
\end{align*}
Definiendo una nueva sustitución:
\begin{align*}
    \frac{X_k}{2^k}&=W_k \xrightarrow{Con:} W_0=0
\end{align*}
Reemplazando con dicha nueva sustitución, se obtiene:
\begin{align*}
    W_k &= W_{k-1} + \frac{1}{3}\left(\frac{3}{2}\right)^k\\
    W_k - W_{k-1}  &= \frac{1}{3}\left(\frac{3}{2}\right)^k \tag{$\odot$}
\end{align*}
Se resuelve la ecuación ($\odot$):
\begin{align*}
    W_k - W_{k-1} &= \frac{1}{3}\left(\frac{3}{2}\right)^k \quad \cdot\text{/}\sumatoria{i=1}{k}\\
    \sumatoria{i=1}{k}\left(W_i - W_{i-1}\right)&=\frac{1}{3}\sumatoria{i=1}{k}\left(\frac{3}{2}\right)^i\\
    W_k - 0 &= \frac{1}{3} \left[ \frac{\left(\frac{3}{2}\right)^{k+1} -\frac{3}{2}}{\frac{3}{2}-1} \right]\\
    W_k &= \frac{2}{3} \left[ \left(\frac{3}{2}\right)^{k+1} - \frac{3}{2} \right]\\
    W_k &= \left( \frac{3}{2} \right)^k -1 \quad; W_k = \frac{X_k}{2^k}\\
    X_k &= 3^k - 2k
\end{align*}
Volviendo a la variable original $X_k=A(n)$ con $n=3^k \xrightarrow{} \log_3(n)=k$, talque:
\begin{align*}
    \therefore A(n) &= n -2^{\log_3(n)}, n=3^k,k \ge 0
\end{align*}
\end{solution}