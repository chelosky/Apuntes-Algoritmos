\question{
Describa la ecuación de recurrencia para evaluar el costo $T(n)$ del siguiente algoritmo, contando los productos realizados (instrucción $s = s*A[i]$).
\begin{lstlisting}
// Ejercicio7_Guia1.java
public void Prueba(int[] A,int n) {
    if(n>1){
        int s=0;
        for(int i=0;i<=n;i++){
            s=s*A[i];
        }
        Prueba(A,n-1);
    }
}    
\end{lstlisting}
}
%\begin{proof}
%\end{proof}

\begin{solution}
Sea $T(n)$ la cantidad de veces que se ejecuta la instrucción $s = s*A[i]$. Además, se sabe que el valor inicial es $T(1)=0$, dado que si $n=1$ entonces la instrucción $s = s*A[i]$ no se ejecuta nunca.

En terminos generales, el termino $T(n)$ ejecuta $n$ veces la instrucción $s = s*A[i]$ (Dado por el ciclo for) y luego se llama recursivamente la misma función pero para el termino anterior $T(n-1)$. Con esto la ecuación de recurrencia resultante es:

$$T(n) = n + T(n-1) \quad , n > 1$$

Resolviendo la ecuación se obtiene:

\begin{align*}
    T(n) - T(n-1) &=  n \quad \text{/}\cdot\sumatoria{i=2}{n} \\
    \sumatoria{i=2}{n}(T(i) - T(i-1)) &= \sumatoria{i=2}{n}i\\
    T(n) -T(0) &= \frac{n(n+1)}{2} - 1\\
    \therefore T(n) &= \frac{n^2 + n- 2}{2} \quad, n\ge 1
\end{align*}

\end{solution}