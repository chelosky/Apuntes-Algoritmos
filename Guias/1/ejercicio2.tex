\question{
Se sabe que la suma de los $n>0$ primeros números impares (a partir de 1) siempre suma el cuadrado de la cantidad de los números sumados. Por ejemplo:
        
    \begin{eqnarray*}
        n = 1 &\rightarrow& 1\\
        n = 2 &\rightarrow& 1 + 3 = 4\\
        n = 3 &\rightarrow& 1 + 3 + 5 = 9
    \end{eqnarray*}
    
Escriba una ecuación de recurrencia para representar este resultado y resuélvala para demostrar este resultado.
}

\begin{solution}
  %\begin{proof}
  %\end{proof}
  \begin{align*}
    a_{n+1} &= a_n + (2n +1) & n \ge 1\\
  \end{align*}
  
  \begin{align*}
      a_n &=a_{n-1} + 2(n-1) + 1 & n\ge 2 \\
      a_n &= a_{n-1} + 2n - 1 \\
      a_n - a_{n-1} &= 2n - 1 & / \cdot \sumatoria{i=2}{n} \\
      \sumatoria{i=2}{n} (a_i - a_{i-1})  &= 2\sumatoria{i=2}{n}i - \sumatoria{i=2}{n}1 \\
      a_n - a_1 &= 2 \left[ \frac{n(n+1)}{2}-1\right] - (n- 1)\\
      a_n - 1 &= n(n+1) -2 -n + 1\\
      a_n &= n^2 +n -n -2 +2 \\
      \therefore a_n &= n^2
  \end{align*}
\end{solution}