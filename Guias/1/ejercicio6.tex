\question{
Los dos primeros términos de una sucesión son 1 y 2. Si se sabe que cada término es la media aritmética del anterior con la media aritmética de los dos adyacentes (anterior y posterior), se pide:
\begin{enumerate}[(a)]
        \item Hallar una fórmula explícita para los términos de la sucesión
        \item Determinar a partir de la fórmula los valores de los términos 3, 4 y 5 de la sucesión
\end{enumerate}
}
%\begin{proof}
%\end{proof}

\begin{solution}
\begin{enumerate}[(a)]
    \item Según lo indicado en el enunciado del ejercicio, los 2 primeros terminos de la sucesión son:
$$C_1=1$$
$$C_2=2$$
Además, cada termino $C_n$ es la media aritmetica del termino anterior $C_{n-1}$ con la media aritmetica de los terminos $C_{n+1}$ y $C_{n-1}$(termino siguiente y anterior respectivamente). Con esto se logra la siguiente ecuación:
\begin{align*}
    C_n &= \frac{C_{n-1} + \frac{C_{n+1}+C_{n-1}}{2}}{2} \\
    C_n &= \frac{2C_{n-1} + C_{n+1} + C_{n-1}}{4} \\
    C_n &= \frac{3C_{n-1} + C_{n+1}}{4} \\
    4C_n &= 3C_{n-1} + C_{n+1} \\
    C_{n+1} -4C_n + 3C_{n-1} &= 0 \quad ,n \ge 2  \tag{$\vartheta$}
\end{align*}
El ejercicio legalmente finaliza aqui, dado que con esto se logra resolver el apartado b del problema. Sin embargo, Recuerde! Usted es informatico! Usted es! INTELIGENTE!(IS THAT A OP REFERENCE!?).

Primeramente, se debe manipular los indices de la ecuación ($\vartheta$), para que solo existan terminos mayores o iguales a n. Para ello se sustituye $n$ por $(n+1)$, obteniendo asi:

\begin{align*}
    C_{(n+1)+1}-4C_{(n+1)}+3C_{(n+1)-1}&=0 \quad ,(n+1) \ge 2 \\
    C_{n+2}-4C_{n+1}+3C_{n}&=0 \quad ,n \ge 1 \tag{$\Psi$}
\end{align*}

La ecuación ({$\Psi$}), se le conoce como una Ecuación Homogénea de primer orden. Para resolverla se asume que existe una solución del tipo $C_n = \lambda^n$ y sustituyendo en ({$\Psi$}) se obtiene:

\begin{align*}
\lambda^{n+2} -4\lambda^{n+1} + 3\lambda^{n} = 0 
\end{align*}


Considerando que $\lambda\neq0$, entonces es posible dividir por $\lambda^n$ en la ecuación ($\Psi'$), tal que:

\begin{align*}
  \lambda^{2} -4\lambda + 3 = 0 \tag{$\Upsilon$}  
\end{align*}

Esta ecuación se llama el polinomio característico vinculado a la ecuación original. Ahora se procede a buscar las raices de la ecuación ($\Upsilon$), tal que:
 
\begin{align*}
\lambda^{2} -4\lambda + 3 &= 0 \\
(\lambda -3)(\lambda-1) &= 0
\end{align*}

Obteniendo asi las raices $\lambda_1=1$ y $\lambda_2=3$, con estas se obtiene la siguiente solución general:

\begin{align*}
    C_n &= K_1 \lambda^n + K_2 \lambda^n \\
    C_n &= K_1 \cdot 1^n + K_2 \cdot 3^n \\
    C_n &= K_1+ K_2 \cdot 3^n \tag{$\Xi$} \\
\end{align*}

Haciendo uso de los valores iniciales $C_1=1$ y $C_2=2$, se obtiene un sistema de ecuaciónes de 2x2, tal que:

$$C_1=1 \rightarrow 1 = K_1 + 3K_2$$
$$C_2=2 \rightarrow 2 = K_1 + 9K_2$$

Donde las soluciones de este sistema son $K_1=\frac{1}{2}$ y $K_2=\frac{1}{6}$. Reemplazando estos valores en ($\Xi$), se obtiene:

\begin{align*}
    C_n &= \frac{1}{2} + \frac{1}{6} \cdot 3^n \\
    \therefore C_n &= \frac{1}{2} (1 + 3^{n-1}) \quad n \ge 1
\end{align*}

    \item \begin{center}
    \begin{tabular}{@{}llllll@{}}
        \toprule
        n & 1 & 2 & 3  & 4  & 5  \\ \midrule
        $C_n$ & 1 & 2 & 5 & 14 & 41 \\ 
        \bottomrule
    \end{tabular}\\
\end{center}
  \end{enumerate}
\end{solution}