\question{
Resuelva la ecuación:
\begin{align*}
    A_n &= 2(A_{n-1} - A_{n-2})\quad, n > 1\\
    A_0 &= 1\\
    A_1 &= 2
\end{align*}
}
%\begin{proof}
%\end{proof}
\begin{solution}
Primeramente observe que la ecuación se puede re-escribir como:
\begin{align*}
    A_n &= 2(A_{n-1} - A_{n-2})\\
    A_n &= 2A_{n-1} - 2A_{n-2}\\
    A_n - 2A_{n-1} + 2A_{n-2} &= 0
\end{align*}
Dejando una ecuación Homogénea de primer orden, para la cual debemos encontrar su polinomio característico:
\begin{align*}
    \lambda^2 -2\lambda +2 &= 0
\end{align*}
La cual tiene como solución una raiz complejo $\lambda_{\star} = \frac{1}{2} \pm \frac{1}{2}i$. 

Recordad, que si tenemos una raiz compleja del tipo $\lambda = a \pm bi$, donde $a,b \in \mathbb{R}$. Entonces, esta puede ser representada en su forma polar como:

$$\lambda = r(\cos{\theta} \pm i\sin{\theta})$$

En donde el radio $r=\sqrt{a^2 + b^2}$ y el angulo $\theta = \arctan\left(\frac{b}{a}\right)$.

En nuestro caso para $\lambda_{\star} = \frac{1}{2} \pm \frac{1}{2}i$, se tiene que su radio es $r = \sqrt{\frac{1}{2}^2 + \frac{1}{2}^2} =\frac{\sqrt{2}}{2}$ y su angulo es $\theta = \arctan\left(\frac{\frac{1}{2}}{\frac{1}{2}}\right)=\frac{\pi}{4}$. Obteniendo asi la siguiente expresion polar:
$$\lambda_{\star} = \frac{\sqrt{2}}{2}(\cos{\left( \frac{\pi}{4} \right)} \pm i\sin{\left( \frac{\pi}{4} \right)})$$

Esta raiz $\lambda_{\star}$ se puede descomponer en $2$ raices, una para la parte imaginaria positiva y otra negativa, tal que:
\begin{align*}
    \lambda_1 &= \frac{\sqrt{2}}{2}(\cos{\left( \frac{\pi}{4} \right)}+i\sin{\left(\frac{\pi}{4} \right)}) \\
    \lambda_2 &= \frac{\sqrt{2}}{2}(\cos{\left( \frac{\pi}{4} \right)}-i\sin{\left(\frac{\pi}{4} \right)}) 
\end{align*}

Aplicando el Teorema de Moivre(Ver Apendice), se obtiene la siguiente solución general:

\begin{align*}
    A_n &= C_1 \cdot \lambda_1^n + C_2 \cdot \lambda_2^n \\
    A_n  &= C_1 \cdot \left(\frac{\sqrt{2}}{2}\right)^n (\cos{\left( n\frac{\pi}{4} \right)}+i\sin{\left(n\frac{\pi}{4} \right)}) + C_2 \cdot \left(\frac{\sqrt{2}}{2}\right)^n(\cos{\left( n\frac{\pi}{4} \right)}-i\sin{\left(n\frac{\pi}{4} \right)}) \tag{$\star$}
\end{align*}
Haciendo uso de los valores iniciales $A_0$ y $A_1$, se obtiene un \st{HORRIBLE} sistema de ecuaciones de 2x2, tal que:
$$
  \left.
    \begin{array}{rcrrcr}
      A_0=1 &\rightarrow& C_1 + C_2 &=&1\\
      A_1=2 &\rightarrow& C_1(\frac{1}{2}+i\frac{1}{2})+C_2(\frac{1}{2}-i\frac{1}{2}) &=&2
      \end{array}
  \right\}
  \begin{array}{rcr}
       C_1&=& \frac{1}{2}(1-3i)  \\
       C_2&=& \frac{1}{2}(1+3i)
  \end{array}
$$
Reemplazando los valores de $C_1$ y $C_2$ en la solución general ($\star$), tal que:
\begin{multline*}
\therefore A_n = \frac{1}{2}(1-3i) \cdot \left(\frac{\sqrt{2}}{2}\right)^n (\cos{\left( n\frac{\pi}{4} \right)}+ i\sin{\left(n\frac{\pi}{4} \right)})\\
 +\frac{1}{2}(1+3i) \cdot \left(\frac{\sqrt{2}}{2}\right)^n(\cos{\left( n\frac{\pi}{4} \right)}-i\sin{\left(n\frac{\pi}{4} \right)}) \quad n\ge0
\end{multline*}
\end{solution}