\question{
Para la ecuación 
\begin{align*}
    T(n) &= 2T\left(\frac{n}{3}\right) - T\left(\frac{n}{9}\right) + 4 \quad, n > 3\\
    T(1) &= 0 \\
    T(3) &= 3
\end{align*}
Resuelva usando un cambio de variable adecuado para n.
}
%\begin{proof}
%\end{proof}
\begin{solution}
En primer lugar debemos determinar que cambio de variable aplicaremos, esto se hace determinando los primeros terminos de la sucesión:
  \begin{center}
    \begin{tabular}{@{}llllll@{}}
        \toprule
        n & 1 & 3 & 9 & 27 \\ \midrule
        T(n) & 0 & 3 & 10 & 21 \\ 
        \bottomrule
    \end{tabular}\\
  \end{center}
  
  Podemos observar que n es múltiplo de 3 ($3^0$ = 1, $3^1$ = 3, $3^2$ = 9,...), por lo tanto el cambio de variable adecuado es:
  \begin{center}
      n = $3^k$, con $k \ge 0$ $\xrightarrow{}$ $\log_3 (n) = k$
  \end{center}
  
  Aplicamos el cambio de variable a la ecuación de recurrencia inicial:
  
  \begin{align*}
      T(n) &= 2T\left(\frac{n}{3}\right) - T\left(\frac{n}{9}\right) + 4 &\quad&, n > 3\\
      T(3^k) &= 2T\left(\frac{3^k}{3}\right) - T\left(\frac{3^k}{9}\right) +4 &\quad&, 3^k > 3\\
      T(3^k) &= 2T\left(\frac{3^k}{3}\right) - T\left(\frac{3^k}{3^2}\right) +4 &\quad&, \log_3 (3^k) > \log_3 (3)\\
      T(3^k) &= 2T\left(3^{k-1}\right) - T\left(3^{k-2}\right) +4 &\quad&, k > 1 \tag{$\star$}
  \end{align*}
  Definimos una nueva sustitución:
  \begin{align*}
      X_k = T\left(3^k\right)
  \end{align*}
  Con los siguientes valores iniciales:
  $$
  \left.
    \begin{array}{rcrrrrlcr}
      n=1 &\implies& T(1)&=&X_0&=&0 &\rightarrow& k=0\\
      n=3 &\implies& T(3)&=&X_1&=&3 &\rightarrow& k=1\\
      n=9 &\implies& T(9)&=&X_2&=&10 &\rightarrow& k=2\\
      n=27 &\implies& T(27)&=&X_3&=&21 &\rightarrow& k=3
    \end{array}
  \right.
  $$
  Reemplazando esta nueva sustitución en la ecuación ($\star$):
  \begin{align*}
      X_k &= 2X_{k-1} -X_{k-2} + 4 \\
      X_k - 2X_{k-1} + X_{k-2} &= 4 \tag{$\dagger$}
  \end{align*}
  Lo que nos deja una Ecuación no Homogénea de primer orden. Por lo cual,  se hace uso del operador $\Delta$, que esta definido como:
$$\Delta F_n = F_{n+1} - F_n$$
Para este ejercicio se aplicara $1$ vez el operador $\Delta$, dado que el mayor exponente de la polinomio del lado derecho de la ecuación ($\dagger$) es $1$.

Una vez aplicado el operador $\Delta$, se busca las raices del polinomio caracteristico asociado a la ecuación  del lado izquierdo de la ecuación ($\dagger$), tal que:

\begin{align*}
    \lambda^2 - 2\lambda + 1  &= 0\\
    (\lambda - 1)^2  &= 0
\end{align*}
Obteniendo asi $2$ raices $\lambda_1 = 1$ y $\lambda_2=1$. Sin embargo, recordad que usamos $1$ vez el operador $\Delta$, se debe añadir $1$ nueva raiz con valor $1$, las cuales definiremos como $\lambda_3 =1$.

Además dado que todas las raices son iguales, debe aplicarse una multplicidad a $\lambda_2$ y una multplicidad doble a $\lambda_3$.

Con estas raices se obtiene la siguiente solución general:

\begin{align*}
    X_k &= C_1 \cdot \lambda_1^k + C_2 \cdot k \cdot \lambda_2^k + C_3 \cdot k^2 \cdot \lambda_3^k\\
    X_k &= C_1 \cdot 1^k + C_2 \cdot k \cdot 1^k + C_3 \cdot k^2 \cdot 1^k\\
    X_k &= C_1 + C_2 \cdot k + C_3 \cdot k^2 \tag{$\star$}
\end{align*}

Como ya deberia haberse percatado, debemos encontrar los valores de $C_1$,$C_2$ y $C_3$ haciendo uso de los valores iniciales $X_0$, $X_1$ y $X_2$. Con esto \st{(y un bizcocho)} obtenemos un sistema de ecuaciones de 3x3, tal que:
$$
  \left.
    \begin{array}{rcrrcr}
      X_0=1 &\rightarrow& C_1 &=&0\\
      X_1=3 &\rightarrow& C_1 +C_2 + C_3 &=&3\\
      X_2=10 &\rightarrow& C_1 + 2C_2+4C_3 &=&10
      \end{array}
  \right\}
  \begin{array}{rcr}
       C_1&=& 0 \\
       C_2&=& 1\\
       C_3&=& 2
  \end{array}
$$
Reemplazando los valores de $C_1$, $C_2$ y $C_3$ en la solución general ($\star$), tal que:
\begin{align*}
    X_k &=k + 2k^2\\
    X_k &=k(1 + 2k)\\
\end{align*}
Volviendo a la variable original con $X_k= T\left(3^k\right)=T(n)$ y $\log_3(n)=k$, tal que:
\begin{align*}
    \therefore T(n)&=\log_3(n)(1+2\log_3(n))\quad n=3^k \text{, } k \ge 0
\end{align*}
\end{solution}