\question{
Para el siguiente algoritmo, sabiendo que $n$ es el tamaño de $L$ y que dividir $L$ tiene un costo total de $n^2$ , y que $T(n)$ es el costo de $IN(L)$:
\begin{lstlisting}
// Ejercicio7_Guia2.java
public int IN(int L) {
    if(L>1){
        Dividir L en tres partes iguales L1, L2, L3;
        IN(L1);
        IN(L2);
        IN(L3);
    }
}    
\end{lstlisting}
Plantear la ecuación de recurrencia para $T(n)$ y resolverla considerando un cambio de variable adecuado.
}
%\begin{proof}
%\end{proof}
\begin{solution}
Sea $T(n)$ la costo del algoritmo $IN(L)$. Sabemos que $T(1)=0$, dado que si $n=1$ entonces la sentencia condicional seria falsa, por lo que no podria ejecutar las demás instrucciones.

En terminos generales, el termino $T(n)$ ejecutara la instrucción de dividir a $L$, $1$ vez con un costo de $n^2$ y además se llama recursivamente $3$ veces a la función $IN()$ pero con los terminos $L1$, $L2$ y $L3$, que tienen un tamaño de $\frac{n}{3}$, dado que son partes iguales de $L$. Con esto la ecuación de recurrencia resultante es:
\begin{align*}
    T(n) &= 3T\left(\frac{n}{3}\right) + n^2 \quad, n >1\\
    T(1) &= 0
\end{align*}
Usando el siguiente cambio de variable:
\begin{align*}
    n=3^k \xrightarrow{} \log_3(n)=k
\end{align*}
Reemplazando dicho cambio en la ecuación de recurrencia inicial:
\begin{align*}
    T(3^k) &= 3T\left(3^{k-1}\right) + 3^{2k} \tag{$\star$}
\end{align*}
Definimos una nueva sustitución:
\begin{align*}
      X_k = T\left(3^k\right)
\end{align*}
Con los siguientes valores iniciales:
  $$
  \left.
    \begin{array}{rcrrrrlcr}
      n=1 &\implies& T(1)&=&X_0&=&0 &\rightarrow& k=0\\
      n=3 &\implies& T(3)&=&X_1&=&9 &\rightarrow& k=1\\
      n=9 &\implies& T(9)&=&X_2&=&108 &\rightarrow& k=2
    \end{array}
  \right.
  $$
Reemplazando $X_k$ en la ecuación ($\star$), tal que:
\begin{align*}
    X_k &= 3 X_{k-1} +3^{2k} \quad, n>0 \tag{$\dagger$}
\end{align*}
Se observa que ($\dagger$) es una Ecuación Lineal con coeficientes constantes, por ello se busca su factor sumante asociado, tal que:
\begin{align*}
    S_k &= \frac{1}{3^k}
\end{align*}
Al multiplicar dicho factor sumante $S_k$, en ambos lados de la ecuación ($\dagger$), se obtiene:
\begin{align*}
    S_k X_k &= 3 S_k X_{k-1} + 3^{2K} S_k \\
    \frac{1}{3^k}X_k &= 3 \frac{1}{3^k} X_{k-1} + 3^{2k} \frac{1}{3^k}\\
    \frac{1}{3^k} X_k &= \frac{1}{3^{k-1}}X_{k-1} + 3^k \tag{$\dagger'$}
\end{align*}
Ahora definamos que:
\begin{align*}
    W_k = \frac{X_k}{3^k} \xrightarrow{con} W_{k-1}=\frac{X_{k-1}}{3^{k-1}}
\end{align*}
Con el siguiente nuevo valor inicial:
\begin{align*}
    W_0 = X_0a_0S_0 = 0\cdot 1\cdot \frac{1}{3^0} = 0
\end{align*}
Haciendo el reemplazo de $W_k$ en la ecuación ($\dagger'$), se obtiene:
\begin{align*}
    W_k &= W_{k-1} +3^k \quad, k>0\\
    W_0
\end{align*}
Resolviendo esta ecuación de recurrencia:
\begin{align*}
    W_k - W_{k-1} &= 3k \quad \cdot \text{/} \sumatoria{i=1}{k}\\
    \sumatoria{i=1}{k} \left(W_i - W_{i-1} \right) &= \sumatoria{i=1}{k}3^i \\
    W_k - W_0 &= \frac{3^{k+1}-3}{2}\\
    W_k &= \frac{3}{2}(3^k -1)
\end{align*}
Volviendo a la termino original con $\frac{X_k}{3^k}=W_k$, tal que:
\begin{align*}
    \frac{X_k}{3^k}&=\frac{3}{2}(3^k-1)\\
    X_k &= \frac{3}{2} 3^k (3^k -1)
\end{align*}
Volviendo a la variable original con $n=3^k \rightarrow X_k=T(n)$, tal que:
\begin{align*}
    \therefore T(n) = \frac{3}{2}n(n-1)\quad ,n=3k \text{ , } k\ge0
\end{align*}
\end{solution}