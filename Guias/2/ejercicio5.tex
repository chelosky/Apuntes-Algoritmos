\question{
Resolver:
\begin{align*}
    X_{n+1} &= X_n +2n +3 \quad n \ge 0\\
    X_0 &= 1
\end{align*}
}
%\begin{proof}
%\end{proof}
\begin{solution}
Primeramente la ecuación original se re-escribir como:
\begin{align*}
    X_{n+1} - X_n&=  2n +3 \tag{$\dagger$}
\end{align*}
Como se vio previamente, este tipo de ecuaciones  se llaman Ecuaciones NO Homogéneas de primer orden. Para resolver este tipo de ecuaciones se hace uso del operador $\Delta$, que esta definido como:
$$\Delta F_n = F_{n+1} - F_n$$
Para este ejercicio se aplicara $2$ veces el operador $\Delta$, dado que el mayor exponente de la polinomio del lado derecho de la ecuación ($\dagger$) es $n$.

Una vez aplicado el operador $\Delta$, se busca las raices del polinomio caracteristico asociado a la ecuación  del lado izquierdo de la ecuación ($\dagger$), tal que:

\begin{align*}
    \lambda -1  &= 0\\
    \lambda &= 1
\end{align*}
Obteniendo asi $1$ raiz $\lambda_1 = 1$. Sin embargo, recordad que usamos $2$ veces el operador $\Delta$, se debe añadir $2$ nuevas raices con valor $1$, las cuales definiremos como $\lambda_2 =1$ y $\lambda_3 = 1$.

Además dado que todas las raices son iguales, debe aplicarse una multplicidad a $\lambda_2$ y una multplicidad doble a $\lambda_3$.

Con estas raices se obtiene la siguiente solución general:

\begin{align*}
    X_n &= C_1 \cdot \lambda_1^n + C_2 \cdot n \cdot \lambda_2^n + C_3 \cdot n^2 \cdot \lambda_3^n\\
    X_n &= C_1 \cdot 1^n + C_2 \cdot n \cdot 1^n + C_3 \cdot n^2 \cdot 1^n\\
    X_n &= C_1 + C_2 \cdot n + C_3 \cdot n^2 \tag{$\star$}
\end{align*}

Como ya deberia haberse percatado, debemos encontrar los valores de $C_1$,$C_2$ y $C_3$ haciendo uso de los valores iniciales $X_0$. Sin embargo, necesitamos encontrar $3$ variables y solo tenemos $1$ valor inicial, por lo que calculamos nuevos valores iniciales para los $2$ siguientes terminos de $n$, es decir $X_1=4$ y $X_2=9$. Con esto \st{(y un bizcocho)} obtenemos un sistema de ecuaciones de 3x3, tal que:
$$
  \left.
    \begin{array}{rcrrcr}
      X_0=1 &\rightarrow& C_1 &=&1\\
      X_1=4 &\rightarrow& C_1 +C_2 + C_3 &=&4\\
      X_2=9 &\rightarrow& C_1 + 2C_2+4C_3 &=&9
      \end{array}
  \right\}
  \begin{array}{rcr}
       C_1&=& 1 \\
       C_2&=&2 \\
       C_3&=& 1
  \end{array}
$$
Reemplazando los valores de $C_1$, $C_2$ y $C_3$ en la solución general ($\star$), tal que:
\begin{align*}
    \therefore X_n &= 1 + 2n + n^2 \quad, n\ge0
\end{align*}
\end{solution}