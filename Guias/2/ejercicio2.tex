\question{
Resolver la ecuación:
\begin{align*}
    T(n) &= 4T(n-3) - 3T(n-1)\quad , n \ge 3\\
    T(0) &= 1\\
    T(1) &= 2\\
    T(2) &= 6
\end{align*}
}
%\begin{proof}
%\end{proof}
\begin{solution}
Primeramente hay que mencionar que la ecuación original se puede escribir como:
\begin{align*}
    T(n) +3 T(n-1) -4T(n-3) &= 0
\end{align*}
Se observa que esta es una Ecuación Homogénea de primer orden. Para resolver esta ecuación se calcula su polinomio característico:
\begin{align*}
    \lambda^3+3\lambda^2-4&=0\\
    (\lambda-1)(\lambda^2 +4\lambda+4)&=0\\
    (\lambda-1)(\lambda+2)^2&=0
\end{align*}
Obteniendo asi las raices $\lambda_1 =1$ y $\lambda_2 = -2$ donde la $\lambda_2$ tiene multiplicidad $2$,obteniendo asi la sigueinte solución general:
\begin{align*}
    T(n) = C_1 \cdot \lambda_1^n + C_2 \cdot \lambda_2^n + C_3\cdot n\cdot \lambda_3^n \\
    T(n) = C_1 \cdot (1)^n + C_2 \cdot (-2)^n + C_3\cdot n\cdot (-2)^n \tag{$\star$}
\end{align*}
Haciendo uso de los valores iniciales $T(0)=1$,$T(1)=2$ y $T(2)=6$, se obtiene un sistema de ecuaciones de 3x3, tal que:
$$
  \left.
    \begin{array}{rcrrcr}
      T(0)=1 &\rightarrow& C_1 + C_2 &=&1\\
      T(1)=2 &\rightarrow& C_1 - 2C_2 -2C_3 &=&2\\
      T(2)=6 &\rightarrow& C_1 + 4C_2+8C_3 &=&6
      \end{array}
  \right\}
  \begin{array}{rcr}
       C_1&=&2  \\
       C_2&=&-1  \\
       C_3&=& 1
  \end{array}
$$
Reemplazando los valores de $C_1$,$C_2$ y $C_3$ en la solución general ($\star$), tal que:
\begin{align*}
    \therefore T(n) &= 2 \cdot (1)^n - (-2)^n + n(-2)^n\quad, n\ge0
\end{align*}
\end{solution}